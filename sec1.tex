\section*{Coordinación internacional} \label{sec:coordinacion}

La organización de las Naciones unidas (ONU), consiente de la problemática relacionada al aumento desmedido de objetos tecnológicos no funcionales en el espacio, creó un comité especializado para coordinar actividades que atiendan la situación;
acciones tanto de monitoreo como protección contra objetos naturales y artificiales.

Este comité, conocido como  \textit{Inter-Agency Space Debris Coordination Committee} o IADC por sus siglas en inglés, agrupa instituciones y países que cuentan con una producción espacial, tales como: Canada, China, Corea, España, Estados Unidos, Francia, India, Italia, Japón, Reino Unido, Rusia y Ucrania.

El objetivo principal de la IADC\cite{iadc} es el intercambio de información sobre las investigaciones realizadas por las agencias miembras con respecto a restos espaciales,
facilitar la cooperación para llevar acabo este tipo de investigaciones, 
analizar el avance en las actividades de cooperación 
e identificar las opciones para mitigar dichos restos.

