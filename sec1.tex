\section{Coordinación internacional} \label{sec:coordinacion}

La organización de las Naciones unidas (ONU), consciente de la problemática relacionada al aumento desmedido de objetos tecnológicos no funcionales en el espacio, creó un comité especializado para coordinar actividades que atiendan la situación;
acciones tanto de monitoreo como protección contra objetos naturales y artificiales.

Dicho comité, conocido como  \textit{Inter-Agency Space Debris Coordination Committee} o IADC por sus siglas en inglés,
tiene como objetivo principal~\cite{iadc_doc} el intercambio de información sobre las investigaciones realizadas por las agencias miembras con respecto a restos espaciales,
facilitar la cooperación para llevar acabo este tipo de investigaciones, 
analizar el avance en las actividades de cooperación 
e identificar las opciones para mitigar los restos.

Las agencias miembras son instituciones que cuentan con una producción espacial. 
Estas se ubican en:  
Canada, China, Corea, España, Estados Unidos, Francia, India, Italia, Japón, Reino Unido, Rusia y Ucrania.

Además, para lograr sus objetivos,
la IADC se compone de un grupo directivo y cuatro grupos de trabajo que cubren:
mediciones (WG1),
medio ambiente y bases de datos (WG2),
protección (WG3) 
y
mitigación (WG4).

En este artículo nos enfocamos en particular en la tareas de observación. 
A continuación discutimos algunas de las redes de vigilancia de residuos espaciales más importantes.
