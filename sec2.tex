\section*{Red de vigilancia espacial de los Estados Unidos}


La \emph{red de vigilancia espacial de los Estados Unidos} (SSN) pertenece al departamento de defensa.
Esta red inició como una combinación de telescopios ópticos y radares en tierra, incorporando más recientemente telescopios en satélites.
Su misión es detectar, rastrear, identificar y catalogar todos los objetos artificiales que orbitan la Tierra, así como proveer información sobre estos a multiples usuarios.\cite{SSN_chatters_2009}
En cuanto a sus actividades,
estas incluyen la generación de catálogos libres y clasificados de todo los objetos espaciales conocidos
(satélites artificiales activos de todo el mundo y residuos espaciales),
la identificación de objetos espaciales determinando su tamaño, forma, así como el movimiento de satélites,
apoyar la orientación de sistemas de armas anti-satélites,
determinar el tiempo en que pasará un satélite sobre un punto deseado en la Tierra,
predecir la trayectoria a seguir por un objeto que entra a la atmósfera de la Tierra y su punto de impacto en caso de sobrevivir la entrada,
y predecir la colisión de objetos en el espacio.\cite{detect_and_track}

