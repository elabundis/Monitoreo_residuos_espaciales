\section*{Introducción}\label{Sec:Introduction}

La basura tecnológica espacial ha tomado un interés público que se ha visto reflejado tanto en la comunidad científica como por los medios de comunicación tradicional. El aumento en el interés a este problema es resultado de un incremento masivo en los viajes espaciales relacionados principalmente con la colocación de satélites para distintas aplicaciones; satélites de comunicación, climatológicos, de investigación del espacio, entre muchos otros. Mientras la puesta en marcha de estos proyectos asiste múltiples necesidades de la sociedad, tales como la comunicación en locaciones remotas, los equipos utilizados en estos también colocan desafíos importantes al terminar su operación tales como el daño a nuevos objetos puestos en órbita e incluso el riesgo para los seres humanos durante su caída a la Tierra.

Dada la relevancia que ha tomado la cantidad de restos no funcionales en el espacio en este momento, la ONU ha tomado la iniciativa de atender esta situación mediante la creación de una división especializada en el tema.
