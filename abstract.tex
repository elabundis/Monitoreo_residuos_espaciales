\begin{small}
\begin{center}
\vspace{9pt}
\textbf{Resumen}    
\end{center}

\begin{adjustwidth}{20pt}{20pt}
\small \noindent En este artículo se hace una discusión sobre los diversos centros que realizan actividades de monitoreo de basura tecnológica espacial; sus tareas, algunos de sus logros y su organización. Al mismo tiempo se presenta un breve recuento histórico sobre el desarrollo de la problemática de la basura espacial y el surgimiento de un ente internacional que pone en marcha estos centros con el objeto de realizar observaciones, evaluar riesgos para la humanidad y establecer reglamentos para las misiones espaciales.
\end{adjustwidth}


\end{small}

